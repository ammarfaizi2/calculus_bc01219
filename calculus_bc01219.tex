\documentclass[12pt]{article}
\usepackage{amsmath}
\usepackage{amssymb}
\usepackage{amsfonts}
\usepackage{nccmath}
\usepackage{cancel}
\usepackage[margin=0.2in]{geometry}
\usepackage[utf8]{inputenc}
\thispagestyle{empty}
\begin{document}
\par\noindent
\begin{fleqn}[1em]

% a
\begin{align*}
\boxed{
\begin{aligned}
  & \textbf{a. } \lim_{x \to \infty} \frac{\ln\left(x^{2020}\right)}{x} \\
  & = \lim_{x \to \infty} \frac{2020 \ln\left(x\right)}{x} \\
  & = 2020 \lim_{x \to \infty} \frac{\ln\left(x\right)}{x}
    = \frac{\infty}{\infty} \text{ (bertemu indeterminate form)} \\
  & \boxed{
    \begin{aligned}
      & \text{Menggunakan aturan L'Hopital} \\
      & \lim_{x \to a} \frac{f(x)}{g(x)} = \lim_{x \to a} \frac{f'(x)}{g'(x)} \\
    \end{aligned}
  } \\
  & \text{sehingga didapat: } \\
  & 2020 \lim_{x \to \infty} \frac{\ln\left(x\right)}{x}
    = 2020 \lim_{x \to \infty} \frac{\frac{d}{dx} \ln\left(x\right)}{\frac{d}{dx} x} \\ ~
  & = 2020 \lim_{x \to \infty} \frac{1/x}{1}
    = 2020 \lim_{x \to \infty} \frac{1}{x} = 2020 \cdot 0 = 0
    \\ & \textbf{by Ammar Faizi}
\end{aligned}
}
\end{align*}

% b
\begin{align*}
\boxed{
\begin{aligned}
  & \textbf{b. } \lim_{x \to \infty} \frac{x^{2020}}{e^x}
    = \lim_{x\to \infty \:}\left(x^{2020}e^{-x}\right)
    = \infty \cdot 0 \text{ (bertemu indeterminate form)} \\
  & \boxed{
    \begin{aligned}
      & \text{Menggunakan aturan L'Hopital} \\
      & \lim_{x \to a} \frac{f(x)}{g(x)} = \lim_{x \to a} \frac{f'(x)}{g'(x)} \\
    \end{aligned}
  } \\
  & \text{sehingga didapat: } \\
  & \lim_{x \to \infty} \frac{x^{2020}}{e^x}
    = \lim_{x \to \infty} \frac{\frac{d}{dx} x^{2020}}{\frac{d}{dx} e^x} \\
  & = \lim _{x\to \infty}\frac{2020 \cdot x^{2019}}{e^x} = \frac{\infty}{\infty}
    \text{ (masih bertemu indeterminate form, maka diturunkan lagi)} \\
  & = \lim _{x\to \infty}\frac{2020 \cdot 2019 \cdot x^{2018}}{e^x} = \frac{\infty}{\infty}
    \text{ (masih bertemu indeterminate form, maka diturunkan lagi)} \\
  & = \lim _{x\to \infty}\frac{2020 \cdot 2019 \cdot 2018 \cdot  x^{2017}}{e^x} = \frac{\infty}{\infty}
    \text{ (masih bertemu indeterminate form, maka diturunkan lagi)} \\
  & = \lim _{x\to \infty}\frac{2020 \cdot 2019 \cdot 2018 \cdot 2017 \cdot x^{2016}}{e^x} = \frac{\infty}{\infty}
    \text{ (masih bertemu indeterminate form, maka diturunkan lagi)} \\
  & = \lim _{x\to \infty}\frac{2020 \cdot 2019 \cdot 2018 \cdot 2017 \cdot 2016 \cdot x^{2015}}{e^x} = \frac{\infty}{\infty}
    \text{ (masih bertemu indeterminate form, maka diturunkan lagi)} \\
  & ... \\
  & \text{diturunkan sebanyak 2020 kali, sehingga tidak menghasilkan indeterminate form lagi} \\
  & ... \\
  & ... \text{ (proses disingkat, hasilnya membentuk pola yang bagus, mudah dibaca)} \\
  & ... \\
  & \text{setelah diturunkan 2020 kali menjadi pola factorial berikut} \\
  & = \lim _{x\to \infty}\frac{
    2020 \cdot 2019 \cdot 2018 \cdot 2017 \cdot 2016 \cdot
    2015 \cdot 2014 \cdot 2013 \cdot ... \cdot
    8 \cdot 7 \cdot 6 \cdot 5 \cdot 4 \cdot 3 \cdot 2 \cdot 1
  }{e^x} \\
  & = \lim _{x\to \infty}\frac{2020!}{e^x} = \frac{2020!}{\infty} = 0
  \\ & \textbf{by Ammar Faizi}
\end{aligned}
}
\end{align*}

% c
\begin{align*}
\boxed{
\begin{aligned}
  & \textbf{c. } \lim_{x \to \pi/2} \frac{3 \sec(x) + 5}{\tan(x)} \\
  & \boxed{\text{Catatan: } \tan(x) = \frac{\sin(x)}{\cos(x)},\:~\:\sec(x) = \frac{1}{\cos(x)}} \\
  & = \lim_{x \to \pi/2}\:\frac{3 \sec(x) + 5}{\sin(x)\:/\cos(x)} \\
  & = \lim_{x \to \pi/2}\:\left(\frac{3 \sec(x)}{\sin(x)\:/\cos(x)} + \frac{5}{\sin(x)\:/\cos(x)}\right)\\
  & = \lim_{x \to \pi/2}\:\left(\frac{3(1/\cos(x))}{\sin(x)\:/\cos(x)} + \frac{5}{\sin(x)\:/\cos(x)}\right)\\
  & = \lim_{x \to \pi/2}\:\left(\frac{3\:\cancel{\cos(x)}}{\cancel{\cos(x)}\sin(x)} + \frac{5\cos(x)}{\sin(x)}\right) \\
  & = \lim_{x \to \pi/2}\:\left(\frac{3}{\sin(x)} + \frac{5\cos(x)}{\sin(x)}\right) \\
  & = \left(\frac{3}{\sin(\pi/2)} + \frac{5\cos(\pi/2)}{\sin(\pi/2)}\right) \\
  & = \frac{3}{1} + \frac{5 \cdot 0}{1} = \frac{3}{1} + 0 = 3
  \\ & \textbf{by Ammar Faizi}
\end{aligned}
}
\end{align*}


% d
\begin{align*}
\boxed{
\begin{aligned}
  & \textbf{d. } \lim_{x \to \infty} [x - \ln(x)] \\
  & = \lim _{x\to \infty}x\left[1-\frac{\ln \left(x\right)}{x}\right] \text{ (dipecah menjadi 2 limit)} \\
  & = \left(\lim _{x\to \infty}x\right) \cdot \left(\lim_{x\to \infty}\left[1 - \frac{\ln(x)}{x}\right]\right) \text{ (bertemu indeterminate form pada limit ke-2)}\\
  & \boxed{
    \begin{aligned}
      & \text{Menggunakan aturan L'Hopital} \\
      & \lim_{x \to a} \frac{f(x)}{g(x)} = \lim_{x \to a} \frac{f'(x)}{g'(x)} \\
    \end{aligned}
  } \\
  & = \left(\lim _{x\to \infty}x\right) \cdot \left(\lim_{x\to \infty}\left[1 - \frac{\frac{d}{dx}\ln(x)}{\frac{d}{dx}x}\right]\right) \\
  & = \left(\lim _{x\to \infty}x\right) \cdot \left(\lim_{x\to \infty}\left[1 - \frac{1/x}{1}\right]\right) \\
  & = \left(\lim _{x\to \infty}x\right) \cdot \left(\lim_{x\to \infty}\left[1 - \frac{1}{x}\right]\right) \\
  & = \left(\lim _{x\to \infty}x\right) \cdot (1 - 0) \\
  & = \infty \cdot 1 = \infty
  \\ & \textbf{by Ammar Faizi}
\end{aligned}
}
\end{align*}


% e
\begin{align*}
\boxed{
\begin{aligned}
  & \textbf{e. } \lim_{x \to 0}\:\left[\csc^2(x) - \frac{1}{x^2}\right] \\
  & \boxed{\text{Catatan: } \csc(x) = \frac{1}{\sin(x)},\:~\:\sin(2x) = 2\sin(x)\cos(x)} \\
  & = \lim_{x \to 0}\:\left[\frac{1}{\sin^2(x)} - \frac{1}{x^2}\right] \\
  & = \lim_{x \to 0}\:\left[\frac{x^2}{x^2\sin^2(x)} - \frac{\sin^2(x)}{x^2\sin^2(x)}\right] \text{ (disamakan penyebutnya)} \\
  & = \lim_{x \to 0}\:\frac{x^2 - \sin^2(x)}{x^2\sin^2(x)}
    = \frac{0^2 - \sin^2(0)}{0^2\sin^2(0)}
    = \frac{0}{0} \text{ (bertemu indeterminate form) } \\
  & \boxed{
    \begin{aligned}
      & \text{Menggunakan aturan L'Hopital} \\
      & \lim_{x \to a} \frac{f(x)}{g(x)} = \lim_{x \to a} \frac{f'(x)}{g'(x)} \\
    \end{aligned}
  } \\
  & \text{sehingga } \lim_{x \to 0}\:\frac{x^2 - \sin^2(x)}{x^2\sin^2(x)}
    = \lim_{x \to 0}\:\frac{\frac{d}{dx}\left(x^2 - \sin^2(x)\right)}{\frac{d}{dx}\left(x^2\sin^2(x)\right)} \\
  & = \lim _{x\to \:0}\frac{2x-\sin \left(2x\right)}{2x\sin ^2\left(x\right)+\sin \left(2x\right)x^2}
    = \frac{0}{0} \text{ (masih bertemu indeterminate form, diturunkan lagi)} \\
  & = \lim _{x\to \:0}\frac{\frac{d}{dx}(2x-\sin \left(2x\right))}{\frac{d}{dx}(2x\sin ^2\left(x\right)+\sin \left(2x\right)x^2)} \\
  & = \lim _{x\to \:0}\frac{2-\cos \left(2x\right)\cdot \:2}{2x^2\cos \left(2x\right)+4x\sin \left(2x\right)+2\sin ^2\left(x\right)} \\
  & = \lim _{x\to \:0} \frac{1-\cos \left(2x\right)}{x^2\cos \left(2x\right)+2x\sin \left(2x\right)+\sin ^2\left(x\right)} = \frac{0}{0} \text{ (masih bertemu indeterminate form, diturunkan lagi)} \\
  & = \lim _{x\to \:0} \frac{\frac{d}{dx}(1-\cos \left(2x\right))}{\frac{d}{dx}(x^2\cos \left(2x\right)+2x\sin \left(2x\right)+\sin ^2\left(x\right))} \\
  & = \lim _{x\to \:0} \frac{2\sin \left(2x\right)}{2x\cos \left(2x\right)-2x^2\sin \left(2x\right)+2\sin \left(2x\right)+4x\cos \left(2x\right)+\sin \left(2x\right)} \\
  & = \lim _{x\to \:0} \frac{2\sin \left(2x\right)}{6x\cos \left(2x\right)+3\sin \left(2x\right)-2x^2\sin \left(2x\right)} = \frac{0}{0} \text{ (masih bertemu indeterminate form, diturunkan lagi)} \\
  & = \lim _{x\to \:0} \frac{\frac{d}{dx}(2\sin \left(2x\right))}{\frac{d}{dx}(6x\cos \left(2x\right)+3\sin \left(2x\right)-2x^2\sin \left(2x\right))} \\
  & = \lim _{x\to \:0} \frac{4\cos \left(2x\right)}{6\cos \left(2x\right)-12x\sin \left(2x\right)+6\cos \left(2x\right)-2\left(2x\sin \left(2x\right)+\cos \left(2x\right)\cdot \:2x^2\right)} \\
  & = \lim _{x\to \:0} \frac{4\cos \left(2x\right)}{12\cos \left(2x\right)-4x^2\cos \left(2x\right)-16x\sin \left(2x\right)} \\
  & = \lim _{x\to \:0} \frac{\cancel{4}\cos \left(2x\right)}{\cancel{4}(3\cos \left(2x\right)-x^2\cos \left(2x\right)-4x\sin \left(2x\right))} \\
  & = \lim _{x\to \:0}\frac{\cos \left(2x\right)}{3\cos \left(2x\right)-x^2\cos \left(2x\right)-4x\sin \left(2x\right)} \\
  & = \frac{\cos \left(2(0)\right)}{3\cos \left(2(0)\right)-0^2\cos \left(2(0)\right)-4(0)\sin \left(2(0)\right)} \\
  & = \frac{\cos(0)}{3\cos(0) - 0 - 0} = \frac{1}{3}
  \\ & \textbf{by Ammar Faizi}
\end{aligned}
}
\end{align*}

% f
\begin{align*}
\boxed{
\begin{aligned}
  & \textbf{f. } \lim_{x \to 0} (\cos(x))^{\csc(x)} \\
  & \boxed{\text{Menggunakan aturan pangkat: } \:a^x=e^{\ln \left(a^x\right)}=e^{x\cdot \ln \left(a\right)}} \\
  & \text{sehingga didapatkan } \cos(x) =  \\
  & \lim_{x \to 0} (\cos(x))^{\csc(x)} = \lim_{x \to 0} \left(e^{\ln\left(\left(\cos(x)\right)^{\csc(x)}\right)}\right) \\
  & = \lim_{x \to 0} \left(e^{\csc(x) \ln(\cos(x))}\right) \\
  & \boxed{
    \begin{aligned}
      & \text{Menggunakan aturan rantai pada limit } \\
      & \mathrm{Jika}\:\lim _{u\:\to \:b}\:f\left(u\right)=L,\:\mathrm{dan}\:\lim _{x\:\to \:a}g\left(x\right)=b,\:\mathrm{dan}\:f\left(x\right)\:\text{kontinu pada}\:x=b \text{, maka } \lim _{x\:\to \:a}\:f\left(g\left(x\right)\right)=L
    \end{aligned}
  } \\
  & \text{Dimisalkan } g\left(x\right)=\csc \left(x\right)\ln \left(\cos \left(x\right)\right)\text{ dan }\:f\left(u\right)=e^u \\
  & \lim_{x \to 0} \csc \left(x\right)\ln \left(\cos \left(x\right)\right) = \infty \cdot 0 \text{ (bertemu indeterminate form)} \\
  & \boxed{
    \begin{aligned}
      & \text{Menggunakan aturan L'Hopital} \\
      & \lim_{x \to a} \frac{f(x)}{g(x)} = \lim_{x \to a} \frac{f'(x)}{g'(x)} \\
    \end{aligned}
  } \\
  & \boxed{
    \begin{aligned}
    & \lim_{x \to 0} \csc \left(x\right)\ln \left(\cos \left(x\right)\right) = \lim_{x \to 0} \frac{\ln \left(\cos \left(x\right)\right)}{\frac{1}{\csc \left(x\right)}}
      = \lim_{x \to 0} \frac{\frac{d}{dx} \ln \left(\cos \left(x\right)\right)}{\frac{d}{dx}\frac{1}{\csc \left(x\right)}} \\
    & = \lim_{x \to 0} \frac{-\tan(x)}{\cos(x)} = \lim_{x \to 0} \frac{-\sin(x)}{\cos(x)\cos(x)}
      = \frac{-\sin(0)}{\cos(0) \cos(0)} = \frac{0}{1 \cdot 1} = 0
    \end{aligned}
  } \\
  & \lim _{u\to \:0}\left(e^u\right) = e^0 = 1 \\ ~ \\
  & \boxed{\therefore \lim_{x \to 0} (\cos(x))^{\csc(x)} = 1}
  \\ & \textbf{by Ammar Faizi}
\end{aligned}
}
\end{align*}

% g
\begin{align*}
\boxed{
\begin{aligned}
  & \textbf{g. } \lim_{x \to \pi/2} \left[\tan(x) - \sec(x)\right] \\
  & = \lim_{x \to \pi/2} \left[\tan(x) - \sec(x)\right] \frac{\tan(x) + \sec(x)}{\tan(x) + \sec(x)} \\
  & = \lim_{x \to \pi/2} \frac{(\tan(x) - \sec(x))(\tan(x) + \sec(x))}{\tan(x) + \sec(x)} \\
  & = \lim_{x \to \pi/2} \frac{\tan ^2\left(x\right)-\sec ^2\left(x\right)}{\tan(x) + \sec(x)} \\
  & \boxed{\text{Identitas Trigonometri: } \tan ^2\left(x\right)-\sec ^2\left(x\right)=-1} \\
  & = \lim_{x \to \pi/2} \frac{-1}{\tan(x) + \sec(x)} \\
  & = \frac{-1}{-\infty} \\
  & = 0
  \\ & \textbf{by Ammar Faizi}
\end{aligned}
}
\end{align*}

% h
\begin{align*}
\boxed{
\begin{aligned}
  & \textbf{h. } \lim_{x \to 0^{+}} \frac{\cot(x)}{\sqrt{-\ln(x)}}
  \\ & \textbf{by Ammar Faizi}
\end{aligned}
}
\end{align*}

\end{fleqn}
\end{document}
